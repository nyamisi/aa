% Options for packages loaded elsewhere
\PassOptionsToPackage{unicode}{hyperref}
\PassOptionsToPackage{hyphens}{url}
\PassOptionsToPackage{dvipsnames,svgnames,x11names}{xcolor}
%
\documentclass[
  12pt,
  a4paper,
  DIV=11,
  numbers=noendperiod]{scrartcl}

\usepackage{amsmath,amssymb}
\usepackage{lmodern}
\usepackage{setspace}
\usepackage{iftex}
\ifPDFTeX
  \usepackage[T1]{fontenc}
  \usepackage[utf8]{inputenc}
  \usepackage{textcomp} % provide euro and other symbols
\else % if luatex or xetex
  \usepackage{unicode-math}
  \defaultfontfeatures{Scale=MatchLowercase}
  \defaultfontfeatures[\rmfamily]{Ligatures=TeX,Scale=1}
\fi
% Use upquote if available, for straight quotes in verbatim environments
\IfFileExists{upquote.sty}{\usepackage{upquote}}{}
\IfFileExists{microtype.sty}{% use microtype if available
  \usepackage[]{microtype}
  \UseMicrotypeSet[protrusion]{basicmath} % disable protrusion for tt fonts
}{}
\makeatletter
\@ifundefined{KOMAClassName}{% if non-KOMA class
  \IfFileExists{parskip.sty}{%
    \usepackage{parskip}
  }{% else
    \setlength{\parindent}{0pt}
    \setlength{\parskip}{6pt plus 2pt minus 1pt}}
}{% if KOMA class
  \KOMAoptions{parskip=half}}
\makeatother
\usepackage{xcolor}
\usepackage[left = 4cm, right = 2.5cm, top = 4cm, bottom =
2.5cm]{geometry}
\setlength{\emergencystretch}{3em} % prevent overfull lines
\setcounter{secnumdepth}{5}
% Make \paragraph and \subparagraph free-standing
\ifx\paragraph\undefined\else
  \let\oldparagraph\paragraph
  \renewcommand{\paragraph}[1]{\oldparagraph{#1}\mbox{}}
\fi
\ifx\subparagraph\undefined\else
  \let\oldsubparagraph\subparagraph
  \renewcommand{\subparagraph}[1]{\oldsubparagraph{#1}\mbox{}}
\fi


\providecommand{\tightlist}{%
  \setlength{\itemsep}{0pt}\setlength{\parskip}{0pt}}\usepackage{longtable,booktabs,array}
\usepackage{calc} % for calculating minipage widths
% Correct order of tables after \paragraph or \subparagraph
\usepackage{etoolbox}
\makeatletter
\patchcmd\longtable{\par}{\if@noskipsec\mbox{}\fi\par}{}{}
\makeatother
% Allow footnotes in longtable head/foot
\IfFileExists{footnotehyper.sty}{\usepackage{footnotehyper}}{\usepackage{footnote}}
\makesavenoteenv{longtable}
\usepackage{graphicx}
\makeatletter
\def\maxwidth{\ifdim\Gin@nat@width>\linewidth\linewidth\else\Gin@nat@width\fi}
\def\maxheight{\ifdim\Gin@nat@height>\textheight\textheight\else\Gin@nat@height\fi}
\makeatother
% Scale images if necessary, so that they will not overflow the page
% margins by default, and it is still possible to overwrite the defaults
% using explicit options in \includegraphics[width, height, ...]{}
\setkeys{Gin}{width=\maxwidth,height=\maxheight,keepaspectratio}
% Set default figure placement to htbp
\makeatletter
\def\fps@figure{htbp}
\makeatother

\KOMAoption{captions}{tableheading}
\usepackage{float}
\floatplacement{figure}{ht}
\renewcommand{\topfraction}{.85}
\renewcommand{\bottomfraction}{.7}
\renewcommand{\textfraction}{.15}
\renewcommand{\floatpagefraction}{.66}
\setcounter{topnumber}{3}
\setcounter{bottomnumber}{3}
\setcounter{totalnumber}{4}
\usepackage{fontspec}
\setmainfont{Times New Roman}
\usepackage[utf8]{inputenc}
\hypersetup{unicode=true,pdfusetitle,bookmarks=true,bookmarksnumbered=true,bookmarksopen=true,bookmarksopenlevel=2,breaklinks=false,backref=false,colorlinks=true,linkcolor=blue}
\usepackage{titlecaps}
\makeatletter
\makeatother
\makeatletter
\makeatother
\makeatletter
\@ifpackageloaded{caption}{}{\usepackage{caption}}
\AtBeginDocument{%
\ifdefined\contentsname
  \renewcommand*\contentsname{Table of contents}
\else
  \newcommand\contentsname{Table of contents}
\fi
\ifdefined\listfigurename
  \renewcommand*\listfigurename{List of Figures}
\else
  \newcommand\listfigurename{List of Figures}
\fi
\ifdefined\listtablename
  \renewcommand*\listtablename{List of Tables}
\else
  \newcommand\listtablename{List of Tables}
\fi
\ifdefined\figurename
  \renewcommand*\figurename{Figure}
\else
  \newcommand\figurename{Figure}
\fi
\ifdefined\tablename
  \renewcommand*\tablename{Table}
\else
  \newcommand\tablename{Table}
\fi
}
\@ifpackageloaded{float}{}{\usepackage{float}}
\floatstyle{ruled}
\@ifundefined{c@chapter}{\newfloat{codelisting}{h}{lop}}{\newfloat{codelisting}{h}{lop}[chapter]}
\floatname{codelisting}{Listing}
\newcommand*\listoflistings{\listof{codelisting}{List of Listings}}
\makeatother
\makeatletter
\@ifpackageloaded{caption}{}{\usepackage{caption}}
\@ifpackageloaded{subcaption}{}{\usepackage{subcaption}}
\makeatother
\makeatletter
\@ifpackageloaded{tcolorbox}{}{\usepackage[many]{tcolorbox}}
\makeatother
\makeatletter
\@ifundefined{shadecolor}{\definecolor{shadecolor}{rgb}{.97, .97, .97}}
\makeatother
\makeatletter
\makeatother
\ifLuaTeX
  \usepackage{selnolig}  % disable illegal ligatures
\fi
\IfFileExists{bookmark.sty}{\usepackage{bookmark}}{\usepackage{hyperref}}
\IfFileExists{xurl.sty}{\usepackage{xurl}}{} % add URL line breaks if available
\urlstyle{same} % disable monospaced font for URLs
\hypersetup{
  pdftitle={AQ 399--Research Project},
  pdfauthor={Nyamisi Peter},
  colorlinks=true,
  linkcolor={blue},
  filecolor={Maroon},
  citecolor={Blue},
  urlcolor={Blue},
  pdfcreator={LaTeX via pandoc}}

\title{AQ 399--Research Project}
\usepackage{etoolbox}
\makeatletter
\providecommand{\subtitle}[1]{% add subtitle to \maketitle
  \apptocmd{\@title}{\par {\large #1 \par}}{}{}
}
\makeatother
\subtitle{Research Proposal Guidelines}
\author{Nyamisi Peter}
\date{11/18/2022}

\begin{document}
\maketitle
\ifdefined\Shaded\renewenvironment{Shaded}{\begin{tcolorbox}[sharp corners, interior hidden, frame hidden, enhanced, borderline west={3pt}{0pt}{shadecolor}, breakable, boxrule=0pt]}{\end{tcolorbox}}\fi

\renewcommand*\contentsname{Table of contents}
{
\hypersetup{linkcolor=}
\setcounter{tocdepth}{3}
\tableofcontents
}
\listoffigures
\listoftables
\setstretch{1.5}
A Research Proposal should be written following \textbf{UDSM} format.
You are required to submit a completed draft of your proposal for the
final review and examination. The research proposal must be your own
original work and should not duplicate any other previously works.

A proposals \textbf{MUST} include the following;

\begin{itemize}
\item
  Title page
\item
  Table of Contents
\item
  Introduction
\item
  Materials and Methods
\item
  Other relevant Information (Financial Budget and time frame)
\item
  References
\end{itemize}

\hypertarget{titlecover-page}{%
\subsection*{Title/cover Page}\label{titlecover-page}}
\addcontentsline{toc}{subsection}{Title/cover Page}

This should contain a proposed title, an institutional information, your
names and name of your supervisor.

\begin{itemize}
\tightlist
\item
  \textbf{Title} -- The title should be concise and clear. From the
  title, the reader should be able to predict fairly accurately what the
  project will be about.
\end{itemize}

The title page does not contain a page number.

\hypertarget{table-of-contents}{%
\subsection*{Table of contents}\label{table-of-contents}}
\addcontentsline{toc}{subsection}{Table of contents}

Provide a list of chapters, sections and subsections together with the
page number in which they are found.

You must also provide;

\begin{itemize}
\item
  \textbf{List of Figures} -- This shows location of all figures in the
  proposed document
\item
  \textbf{List of Tables} -- This shows location of all tables in the
  proposed document
\end{itemize}

Table of contents must have a roman page number.

\hypertarget{introduction}{%
\subsection{Introduction}\label{introduction}}

This section must include the following and must have Arabic page
numbers;

\hypertarget{general-introduction-min.-2-pages.}{%
\subsubsection{General introduction (Min. 2
pages).}\label{general-introduction-min.-2-pages.}}

In this subsection explain the general background information about your
research topic. Summarize key available literature and cite the most
current and important previous studies that are relevant to the current
research. In this section, the following must be included;

\begin{itemize}
\item
  Explain what has been done in the past research
\item
  Show what has not been done -- \textbf{the gap}
\item
  Address what are you going to do to reduce or fill the gap
\end{itemize}

\hypertarget{statement-of-research-problem-max.-1-page.}{%
\subsubsection{Statement of Research Problem (Max. 1
page).}\label{statement-of-research-problem-max.-1-page.}}

In this section deals with identification of the problem

The section should include;

\begin{itemize}
\item
  A clear statement that the problem exists
\item
  Evidence that supports the existence of the problem
\item
  Evidence of an existing trend that has led to the problem
\item
  The causes related to the problem
\end{itemize}

\hypertarget{objectives-of-the-research}{%
\subsubsection{Objectives of the
research}\label{objectives-of-the-research}}

This outline where the project is headed and what will be accomplished.
Objectives should directly address the problem mentioned in the problem
statement. They should be very specific. It includes both general and
specific objectives.

\hypertarget{general-objective}{%
\paragraph{General objective}\label{general-objective}}

The general objective is what you want to achieve in your project. This
is the main goal or aim of your research project.

\hypertarget{specific-objective}{%
\paragraph{Specific objective}\label{specific-objective}}

The specific objectives are the building blocks of the general
objective. These are activities which will be done in your research.

\hypertarget{hypothesesresearch-questions}{%
\subsubsection{Hypotheses/Research
questions}\label{hypothesesresearch-questions}}

When writing your proposal, you may either use hypothesis or research
question. A hypothesis is an assumption that is made based on some
evidence of your research. A research hypothesis is the one which will
be tested during data analysis.

In scientific research, hypotheses are normally stated in
\textbf{\emph{NULL}} and not in Alternative hypothesis

A research question is a question that a study or research project aims
to answer. This question often addresses an issue or a problem, which,
through analysis and interpretation of data, is answered in the study's
conclusion.

Every specific objective, must have its own hypothesis or research
question

\hypertarget{significance-of-the-study}{%
\subsubsection{Significance of the
study}\label{significance-of-the-study}}

Significance of the study refers to the importance of your research.
While stating the significance, you must highlight the contribution or
benefits of your research in science and in society. Start with broader
importance, and narrowing it down to demonstrate the specific group that
will benefit from your research.

\hypertarget{literature-review-min.-3-pages}{%
\subsubsection{Literature Review (Min. 3
pages)}\label{literature-review-min.-3-pages}}

A literature review is a comprehensive summary of previous research on a
topic. The literature review surveys scholarly articles, books, and
other sources relevant to a particular area of research. The review
should enumerate, describe, summarize, objectively evaluate and clarify
this previous research. It should give a theoretical base for the
research and help you (the author) determine the nature of your
research. The literature review acknowledges (citing) the work of
previous researchers. It is assumed that by mentioning a previous work
in the field of study, that the author has read, evaluated, and
assimilated that work into the work at hand.

\hypertarget{materials-and-methods}{%
\subsection{Materials and Methods}\label{materials-and-methods}}

A research proposal should contain this section which gives details on
the materials and methods proposed to be used when conducting the
research. The materials are tools, raw materials, subject and/or
important chemicals used in your experiments. Basically these are
important details of \textbf{WHAT} you use in your research. The methods
is \textbf{HOW} you conduct the research. It describe all the steps or
procedures you have done in order to achieve the research objectives,
including the experimental design and data analysis.

This section must include the following;

\hypertarget{study-site-or-area}{%
\subsubsection{Study site or area}\label{study-site-or-area}}

This describe the location(s) where the proposed research will be
carried out. The details of the conditions and geographical
characteristics and the map of the an area should also be included in
this section.

\hypertarget{experimental-design-or-sampling-design}{%
\subsubsection{Experimental design or sampling
design}\label{experimental-design-or-sampling-design}}

This is the step by step descriptions of where and how the experiment or
the study will be conducted. This is the setup of your research
experiment. Explain how you will design your study or experiment, how
often (frequency) will data be collected and for how long you will be
collecting data. This section should be described precisely and in
details; it must be reproducible. If questionnaires will be used,
samples of the proposed questionnaires should be presented.

\hypertarget{data-collection}{%
\subsubsection{Data collection}\label{data-collection}}

Explain the type of data which you will collect. Show the technique will
be used in collect data. If you are using a method that has been
validated; state the method and cite the author. In case new methods or
technique have been developed; describe it in details.

\hypertarget{data-analysis}{%
\subsubsection{Data analysis}\label{data-analysis}}

Data analysis is a process of cleaning, transforming, presenting,
visualizing and modeling data to discover useful information for
decision-making. It involves statistical analyses in order to draw
conclusion based on your research objectives and hypotheses. There are
two types of statistical analysis -- \textbf{Descriptive statistics} and
\textbf{Inferential statistics}

Descriptive statistics is a describes nature of the data or gives a
summary of the data. It shows distribution and variation of the data.
However, you can not make conclusion based on this type of statistics.

Inferential statistics is the type of statistics used to draw
conclusion. There are different types of inferential statistics; some
are \textbf{parametric} such as t-test, ANOVA while some are
\textbf{non-parametric} such as Man-whitney test and Kruskal-wallis.

This section is very important when preparing your research proposal.
State the type of statistical test which you will use to test each
hypothesis. The type of statistical software which you are going to use
during analysis also have to be shown in this section.

\hypertarget{other-relevant-information}{%
\subsubsection{Other relevant
information}\label{other-relevant-information}}

\hypertarget{work-plan-or-timeframe}{%
\paragraph{Work plan or Timeframe}\label{work-plan-or-timeframe}}

A work plan is a schedule, chart or graph that summarizes different
activities of a research project. Every activity of your research have
to be included in this section from formulation of the research title to
the final submission of the report.

\hypertarget{budgeting-and-financial-arrangements}{%
\paragraph{Budgeting and Financial
arrangements}\label{budgeting-and-financial-arrangements}}

Financial budget will include all the cost that will be needed to
conduct and complete the study. Total amount of cost should be broken up
into specific amount.

\hypertarget{references}{%
\subsection{References}\label{references}}

The reference section must contain an alphabetical list of all
references cited in the text. \textbf{Use Tanzania Journal of Science}
referencing style to cite and list the references. All references cited
in the document \textbf{MUST} appear in the reference list, and all in
the reference list MUST be cited in the document.



\end{document}
